\section{Introduction}
In recent years, the exponential growth of graph data scales and the widespread adoption of graph-based models have positioned graph computing accelerators 
and system design as critical research areas attracting significant attention from both academia and industry. 
This surge in interest reflects the fundamental role that graph processing has assumed in big data analytics, 
where traditional computational paradigms struggle to efficiently handle the irregular data structures and complex traversal patterns inherent in graph workloads. 
The unique characteristics of graph algorithms, including their inherently data-dependent control flow and sparse memory access patterns, 
present substantial challenges for achieving high performance on conventional computing architectures.

However, the runtime computational and memory access behaviors of graph computing applications on emerging computing platforms differ 
significantly from those observed on classical processor architectures such as CPUs and GPUs. 
While existing graph computing frameworks provide valuable insights and design principles, 
they cannot be directly applied to custom hardware platforms without substantial modifications, 
thereby limiting the ability to fully exploit the potential of specialized accelerators. 
On the other hand, current graph computing frameworks are typically designed for specific graph domains, 
making them inadequate for addressing the diverse and complex applications encountered in real-world scenarios. 
These limitations underscore the urgent need for novel programming models and compilation techniques 
that can bridge the gap between high-level graph algorithm specification and efficient execution on heterogeneous computing platforms.